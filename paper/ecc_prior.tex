\documentclass[prd,
               reprint,
               letterpaper,
               groupedaddress,
               linenumbers
              ]{revtex4-1}

\usepackage{graphicx}
\usepackage{color}
\usepackage{amsmath,amssymb}
\usepackage{acronym}
\usepackage{hyperref}


\usepackage{blindtext}

%%%%%%%%%%%%%%%%%%%%%%%%%%%%%%%%%%%
%% EDITOR NOTES:
\newcommand{\todo}{\textcolor{red}}
\newcommand{\CiteNeed}{\textcolor{red}{\scriptsize[citation needed]}}
\newcommand{\CheckMe}{\textcolor{orange}{\scriptsize[check me]}}
\renewcommand{\check}{\textcolor{orange}}

%%%%%%%%%%%%%%%%%%%%%%%%%%%%%%%%%%%
%% BETTER DATE:
\renewcommand{\today}{\number\day\space\ifcase\month\or
  January\or February\or March\or April\or May\or June\or
  July\or August\or September\or October\or November\or December\fi
  \space\number\year}

%%%%%%%%%%%%%%%%%%%%%%%%%%%%%%%%%%%
%% ACRONYMS
\def\gw{gravitational wave}
\def\gws{gravitational waves}
\def\GW{Gravitational Wave}
\def\GWs{Gravitational Waves}
\newacro{SNR}{signal-to-noise ratio}
\def\SNR{\ac{SNR}}
\newacro{BNS}{binary neutron star}
\def\BNS{\ac{BNS}}
\newacro{BBH}{binary black hole}
\def\BBH{\ac{BBH}}


\begin{document}

%%%%%%%%%%%%%%%%%%%%%%%%%%%%%%%%%%%
%% FRONT MATTER
\title{Detecting Highly-Eccentric Binaries with a Gravitational Wave Burst Search}

\author{Paul T.~Baker}
\email[]{paul.baker1@mail.wvu.edu}
\author{Belinda~Cheeseboro}
\author{Sean T.~McWilliams}
\affiliation{Center for Gravitational Waves and Cosmology, West Virginia University, Morgantown, WV 26505, USA}
\affiliation{Department of Physics and Astronomy, West Virginia University, Morgantown, WV 26505, USA}
%\altaffiliation{}

\date{\today}

\begin{abstract}
This is an abstract.
Hightly eccentric binaries produce distinct bursts at pericenter crossing.
Connect individual bursts with a prior on their time and frequency content.
\end{abstract}

% insert suggested PACS numbers in braces on next line
\pacs{04.30.-w, 04.80.Nn, 07.05.Kf, 95.55.Ym}

\maketitle


%%%%%%%%%%%%%%%%%%%%%%%%%%%%%%%%%%%
%% Introduction
\section{Introduction}
\label{sec:intro}

Gravitational waves
We are now post GW150914 \cite{GW150914}.
There are lots of \BBH~detections.
\par

Binaries are thought to circularize before entering LIGO band \CiteNeed.
Rodriguez et al. says fraction of \BBH~will have $e\gtrsim0.1$ when they enter band \CiteNeed.
Highly eccentric: dynamical capture, N-body.
\par

Eccentric waveform modeling is hard.
See recent work from Yunes group, Hinder+ 2018, Yang+ 2018... \CiteNeed
So templated searches are hard.
Existing searches with eccentric templates.
\par

Burst searches!
BayesWave \cite{cl2015}.
Targeted bursts: chirplets, cWB's BBH search \CiteNeed.
cWB's eccentric method \CiteNeed.
\par





%%%%%%%%%%%%%%%%%%%%%%%%%%%%%%%%%%%
%% Waveform Model
\section{Waveform Model}
\label{sec:model}

Our model is built from the \textit{Newtonian Burst Model} defined in section 2.1 of Loutrel and Yunes \cite{ly2017}.

The waveform is wavelets.
In order to associate disconnected wavelets we use prior.

The prior is a function of physical meta-parameters that discribe the binary orbit.

each `blob' is defined by a bivariate normal distribution, centered at $(t,f)$.
The covariance is defined by the uncertainty on the centroid location \textit{not} the time-frequency shape of the burst.

Prior knowledge of the duration or bandwidth of a burst would inform the quality factor of the wavelets.



%%%%%%%%%%%%%%%%%%%%%%%%%%%%%%%%%%%
%% Recovery of Signals / Meta-parameters
\section{Recovery of Signals}
\label{sec:recov}
\input{sec_03_recov}


%%%%%%%%%%%%%%%%%%%%%%%%%%%%%%%%%%%
%% Discussion
\section{Discussion}
\label{sec:discuss}
\input{sec_03_recov}


% Specify following sections are appendices. Use \appendix* if there
% is only one appendix.
%\appendix
%\section{}

% If you have acknowledgments, this puts in the proper section head.
%\begin{acknowledgments}
% put your acknowledgments here.
%\end{acknowledgments}

% Create the reference section using BibTeX:
\bibliography{biblio.bib}

\end{document}

